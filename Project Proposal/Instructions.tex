###########################################################################
#                                                                         #
#                          iSci Document Format                           #
#                           Latex Instructions                            #
#                                                                         #
# v1.0 - 01.20.2022: First Published                                       #
# v1.1 - 02.17.2022: Revised Version                                      #
#                                                                         #
###########################################################################

[0] Description of files in this folder

    Example article produced from "iSci Document Format"
    -----------------------------------------------------------
    main.tex - LaTeX source file (Encoding: UTF-8)
    Example.bib - BibTeX source file (Encoding: UTF-8)
    Black Hole Image.jpg - jpg file of an example of a figure

[1] Description of the iSci Document Format Template
    
    This template was created for the Integrated Science program 
    at McMaster University in Hamilton, Ontario, Canada. It it used 
    by students in the program for research project submissions, lab
    reports, and assignments, but it can be used by anyone looking for
    a LaTeX template to script a document. See usage instructions below.

[2] Usage of the iSci Document Format Template

    Below you will find instructions for different typesetting habits.
    Please choose the one suits best for you and apply.

    PERSONAL USAGE
    ----------------
    If you would like to use the template personally please continue
    reading on [A] or [B].

[A] For those who know little to nothing about LaTeX
----------------------------------------------------

    1) If you are completely new to LaTeX, then we are
    hoping that this template will act as you "LaTeX 
    training wheels" in a sense. We have designed it 
    aiming to make the transition from other word  
    processors to LaTeX as easy as possible. We have 
    also included examples for inserting equations,
    figures, and bibliographies. 
    
    2) If you find the learning curve of learning Latex
    a bit challenging, we have attached some links below
    to some useful sources that should help you getting 
    started, or as a reference when scripting
    
    https://en.wikibooks.org/wiki/LaTeX/Basics - A good overview of the basics
    
    https://latex.guide/ - Useful for searching for mathematical symbols
    
    https://www.overleaf.com/learn - Teaches a wide variety of the ins and outs of LaTeX
    
    https://stackoverflow.com/ - A site with a community of developers; good for niche questions
    
    https://www.mn.uio.no/ifi/tjenester/it/hjelp/latex/biblatex-guide.pdf - Good for if you want to modify the citation style
    
    These sources are excellent for making the task of
    learning LaTeX much less daunting. After some practise,
    we hope that you will only have to focus on the writing
    and not on "will this compile?". After all, the importance
    is on WHAT you write, not HOW you write it.
    
    Once you are familiar with these sources, we recommend that 
    you rename the main.tex to the name of your document and 
    change the title from "iSci Document Format" to the title 
    of your work. Once this is done, you are ready to start
    scripting!
    
[B] Quick Start - For those who have used LaTeX in the past
-----------------------------------------------------------
    1) The first step is the change the title of the 
    document from "iSci Document Format" to the title 
    of your work and you rename the main.tex to the 
    name of your document as well.
    
    2) Import any figures into your LaTeX directory by
    clicking upload at the top left of the screen, under
    Menu, and importing the figures from your files.
    
    3) Go the the main tex document and edit using your
    normal typesetting habits and tools. Be sure to change
    any unwanted filler text and example equations/figures
    so they are not present in your final work (Unless of 
    course, you want a black hole image or the time dependent 
    Schrodinger equation/dose attenuation formulas; the figure
    is under the Creative Commons license).
    
    4) Citations can be done using a variety of techniques.
    Zotero can be linked to LaTeX, but this requires a premium
    membership. However, it can still be done using the free 
    version. Once all of your citations are in your Zotero 
    folder, right click on the folder, then click "Export
    Collection". This then takes you to a small window. 
    Here, click export notes. Be sure that it is in BibTex
    format and unicode (UTF-8) character encoding. Then 
    name the BibTex file to whatever you want and download it.
    Now, upload it to the directory as you did for the
    figures. Lastly and very importantly, in the main tex, 
    change the argument in the command \addbibresource{Example.bib}
    on line 58 to contain the name of you bib file. Now
    BibTex should be working for you and it will default to 
    Anglia Ruskin Harvard style. To insert citations at the 
    end of sentences, use the command \autocite{}. To insert
    citations in-text, use \textcite{}. To insert your
    bibliography at the end, use \printbibliography. 
    
    *Note for books, be sure to include editors by adding another author in Zotero, then changing title from author to editor. A simpler but more crude method would involve simply adding the editors directly in the .bib file in the same format as the authors.  
    
    Alternatively, you can export the Zotero collection
    using a word processor like you normally would, however
    you may need to make a list for the bibliography using the commands:
    \begin{itemize}
        \item[] bibliography entry
    \end{itemize}
    
    This can also present challenges in terms or tracking citations 
    if you use a numeric style such as Vancouver, but it should be fine
    when using Anglia Ruskin Harvard.
    
    5) Once your citations are complete, export your
    document from the LaTeX source file. It will be 
    downloaded as a pdf. Before you export, be sure
    to check the following:
     
     (a) Are all author names?
     (b) Are all figures and tables included and formatted correctly?
     (c) Are all captions numbered in correct order?
     (d) Are all cited work (references) listed at the end? Are they cited correctly?
     (e) Is your document free of grammatical errors?
     
     
    We hope you find this template helpful and happy
    scripting!