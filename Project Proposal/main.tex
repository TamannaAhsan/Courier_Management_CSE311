\documentclass[12pt]{article}
\usepackage[utf8]{inputenc}
\usepackage{graphicx} % Allows you to insert figures
\usepackage{amsmath} % Allows you to do equations
\usepackage{fancyhdr} % Formats the header
\usepackage{geometry} % Formats the paper size, orientation, and margins
\linespread{1.25} % about 1.5 spacing in Word
\setlength{\parindent}{0pt} % no paragraph indents
\setlength{\parskip}{1em} % paragraphs separated by one line
\usepackage[style=authoryear-ibid,backend=biber,maxbibnames=99,maxcitenames=2,uniquelist=false,isbn=false,url=true,eprint=false,doi=true,giveninits=true,uniquename=init]{biblatex} % Allows you to do citations - does Harvard style and compatible with Zotero
\urlstyle{same} % makes a nicer URL and DOI font 
\AtEveryBibitem{
    \clearfield{urlyear}
    \clearfield{urlmonth}
} % removes access date
\AtEveryBibitem{\clearfield{month}} % removes months in bibliography
\AtEveryCitekey{\clearfield{month}} % removes months in citations
\renewbibmacro{in:}{} % Removes the "In" before journal names

\renewbibmacro*{editorstrg}{%from biblatex.def
  \printtext[editortype]{%
    \iffieldundef{editortype}
      {\ifboolexpr{
         test {\ifnumgreater{\value{editor}}{1}}
         or
         test {\ifandothers{editor}}
       }
         {\bibcpstring{editors}}
         {\bibcpstring{editor}}}
      {\ifbibxstring{\thefield{editortype}}
         {\ifboolexpr{
            test {\ifnumgreater{\value{editor}}{1}}
            or
            test {\ifandothers{editor}}
          }
            {\bibcpstring{\thefield{editortype}s}}%changed
            {\bibcpstring{\thefield{editortype}}}}%changed
         {\thefield{editortype}}}}}

\renewbibmacro*{byeditor+others}{%from biblatex.def
  \ifnameundef{editor}
    {}
    {\printnames[byeditor]{editor}%
     \addspace%added
     \mkbibparens{\usebibmacro{editorstrg}}%added
     \clearname{editor}%
     \newunit}%
  \usebibmacro{byeditorx}%
  \usebibmacro{bytranslator+others}}
  % The commands above from lines 20-49 change the way editors are displayed in books
\AtEveryBibitem{%
  \clearlist{language}%
} % removes language from bibliography
\citetrackerfalse 
% Removes ibids (ibidems)
\DeclareNameAlias{sortname}{family-given} % Ensures the names of the authors after the first author are in the correct order in the bibliography
\renewcommand*{\revsdnamepunct}{} % Corrects punctuation for authors with just a first initial
\addbibresource{Example.bib} % Tells LaTeX where the citations are coming from. This is imported from Zotero
\usepackage[format=plain,
            font=it]{caption} % Italicizes figure captions
\usepackage[english]{babel}
\usepackage{csquotes}
\renewcommand*{\nameyeardelim}{\addcomma\space} % Adds comma in in-text citations
\renewcommand{\headrulewidth}{0pt}
\geometry{letterpaper, portrait, margin=1in}
\setlength{\headheight}{14.49998pt}

 %%%%% Put your group name here. If you are the only member of the group, just put your name

\begin{document}
\be

\setcounter{page}{1}
\pagestyle{fancy}
\fancyhf{}
\rhead{\thepage}
\lhead{\GroupName; \titleofdoc}

\section*{Introduction} % If you want numbered sections, remove the star after \section

The year is 2022 and everything is online based. Even people are getting attracted to online shopping rather than spending hours and hours roaming around market to buy something. Here comes the big necessity of courier system which can deliver their goods from directly seller to customers. There are already many courier services in the market. But none of them are good enough as they provide low quality services when the expected product either gone missing or they are found broken reaching at the customers hand. This web application we are building will be free from these bad customer experiences as this application will keep track of the package also track the carrier’s information for safe delivering. Also, it will provide customer’s reviewing system.

\section*{Objectives}

There will be two modules. 1. Users & 2. Admin.

1.	User can register and login to the website and also can check their own profile.

2.	Users can see their parcel histories. 

3.	Users will get notifications when their package will be delivered.

4.	Users can send emails to the website if they have any problems.

5.	The website will be controlled by the admin.

6.	Admin can add new parcel orders from the sellers and can also cancel them.

7.	Admin can check the orders are delivered to the users.

8.	Admin can add new admins, delete admins, update their own profile.

9.	Admin can also check emails of users.

10.	Tracking the user’s location.

\setcounter{page}{1}
\section*{Targeted Users}
1.	Admin: Admin will collect delivery orders from the sellers and deliver the products to the users. 

2.	User: Users will receive their ordered products from the website.

\section*{Value Proposition}
This courier system will be a good choice for the online sellers as they won’t have to deliver their products to the customers themselves. Hence this will reduce their hassles and their time will be saved also they can concentrate better on their selling. Customers will also get benefited from this as they don’t have to ask their seller of when they will get their products again and again. They will be able to track their package without asking to anyone from the webapp. This app will be used for easing both customers and sellers buying and selling experiences. This will create some job opportunities for some unemployed peoples where they will work as deliveryman.

\section*{Web Application Features and Description}
This application will open with the general view at first. The goal of this project is to deliver products to the users. First of all, admin will collect orders from the sellers to deliver the products to the customers. Then admin will track the location of the users and send their products to their location. Admin can manage the whole process of adding, cancelling and managing orders. Users can also login to this website. They can check their profile and previous histories as well. They can contact through mail to this website. They will get notification after getting the products.

\section*{Tools and Resources}
•	HTML

•	CSS

•	Bootstrap

•	JavaScript

•	MySQL

•	PHP

We will use HTML, CSS and Bootstrap for Frontend and for Backend we will use PHP and MySQL. 

\section*{Challenges}
To make this project we will face some problems. As we are not familiar with PHP and JavaScript, we need to learn these languages first to start our project. To build this web application we need to learn both frontend and backend part. We also need API services for this system. Also, for proper maintenance in our project we constantly keep this system up to date. Second challenge will be the verification of data. We will need to check if the data entered are reliable or not. Although Bangladesh has started digital NID cards but to access the database will not be easy, many processing needs to be done. UI also need to be simple and easy so that normal people can also use it. Lastly, connection or providing security for online payment will be difficult.


\pagebreak

\printbibliography % If something looks strange in the bibliography, more often than not, you can modify the parameter in the .bib to fix the problem
\end{document}